\documentclass{article}
\usepackage{graphicx}
\usepackage{setspace}
\setlength{\parindent}{0pt}
\spacing{1.5}

\begin{document}

\title{eRm package (Reinhold Hatzinger)}
\author{Salvador Garcia}

\maketitle
\tableofcontents

\section{Introduction}
The eRM package uses Conditional Marginal Likelihood as the method of estimation. 

IRT tries to find a function that maps a latent variable (sometimes expressed as the ability, attitude or trait of the subject) to the probability of a reaction given this latent variable. 

\section{The Rasch model (RM)}

\subsection{The Rasch model}
\begin{equation}
  P(X_{vi} = 1 | \theta_v, \beta_i) = \frac{\exp(\theta_v - \beta_i)}{1+\exp(\theta_v - \beta_i)}
\end{equation}

\begin{itemize}
\item{$X$ is the matrix that the rows represent the persons and the columns the item of the quiz.}
\item{$X_{vi} \in \{0,1\}$ is the entry $(v,i)$ of the matrix $X$. This entry is a binary variable that represents the answer to the item $i$ of the person $v$. $0$ means that the person answers incorrectly and $1$ that answers correctly.}
\item{$\theta_v \in (-\infty, \infty)$ is the ability or trait of the person $v$.}
\item{$\beta_i \in [0, \infty)$ is the difficulty of the item $i$.}
\end{itemize}

It is useful to define two \textit{Raw Scores}. These scores are simple marginalization of $X_{vi}$ with respect to the variables $v$ and $i$:

\begin{itemize}
\item{\textbf{Raw score per person:} $r_v = \sum_i X_{vi}$}
\item{\textbf{Raw score per item:} $s_i = \sum_v X_{vi}$}
\end{itemize}

\subsection{Assumptions of the Rasch Model}
\begin{itemize}
\item{\textbf{Unidimensionality}}
\item{\textbf{Sufficiency}}
\item{\textbf{Conditional independence}}
\item{\textbf{Monotonicity}}
\end{itemize}

\subsection{Item Parameter Estimation}

\subsubsection{Item parameter estimation}

Different methods of estimation for \textbf{item parameter estimation}: 

\textbf{Joint Maximum likelihood Estimation (JML)}
(look for demonstration)
\begin{equation}
L_u = \frac{\exp(\sum_v \theta_v r_v) \exp(\sum_i \beta_i s_i)}{\prod_v \prod_i (1 + \exp(\theta_v - \beta_i)}
\end{equation}

In the last expression we can find that the sufficient statistics are $r_v = \sum_i X_{vi}$ for $\theta_v$ and $s_i = \sum_v X_{vi}$ for $\beta_i$.

Problem: inconsistency of the parameters when $n \rightarrow \infty$

\textbf{Marginal Maximum likelihood Estimation (MML)}

If we integrate the person parameter, we can marginalize the expression:

\begin{equation}
L_m = \prod_r [ \exp(-\sum_i \beta_i s_i) ] continue
\end{equation}


\textbf{Conditional Maximum likelihood Estimation (CML)}

Condition on $r_v$
\begin{equation}
L_c = \exp()
\end{equation}

\textbf{Advantages of CML and MML}


\subsubsection{Person parameter estimation}
Different methods of estimation for \textbf{person parameter estimation}:

\textbf{Maximum Likelihood (ML) and Weighted Maximum Likelihood (WML)}
\\
\textbf{Bayes approach.}


\section{The plots}

\subsection{The PI map}

\subsection{The ICC plot}

\subsection{The joint ICC plot}



\appendix

\section{The eRm package}
The eRm package


\end{document}

